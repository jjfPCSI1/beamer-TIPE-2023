% Le titre de la partie
\section[Relativité générale]{Le point de vue de la Relativité Générale}

%%%%%%%%%%%%%%%%%%%%%%%%%%%%%%%%%%%%%%%%%%%%%%%%
% Première diapo
%%%%%%%%%%%%%%%%%%%%%%%%%%%%%%%%%%%%%%%%%%%%%%%%

\begin{frame}
	\frametitle{Relativité Générale}
	\framesubtitle{Et zut...}

	\begin{alertblock}{Arg...}
		Il n'y a plus guère de temps pour parler...

		Pour voir le code informatique, cliquez sur le bouton \beamergotobutton{\hyperlink{code}{Source}}
	\end{alertblock}

	Alors zou, on balance le résultat des calculs
	$$
	    \onslide<3->
	    \underbrace{
	    \onslide<2->
	    \frac{\dd^2 u}{\dd \theta^2} + u
	        =
	        \frac{G M m^2}{L^2}
	    \onslide<3->
	    }_{\text{Partie classique}}
	    \onslide<2->
	    +
	    \onslide<3->
	    \underbrace{
	    \onslide<2->
	    \frac{3\, G M}{c^2}\, u^2
	    \onslide<3->
	    }_{\text{RG}}
	    \onslide<2->
	$$
	\onslide<+->{
	Après avoir bien bossé, on obtient finalement
	$$
		\boxed{
			\delta = 6\, \pi \, \frac{G M}{a\, c^2\pa{1-e^2}}
				}
	$$
	}
\end{frame}
