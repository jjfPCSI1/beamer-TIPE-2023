% On découpe ce document complexe en plusieurs sous-fichiers séparés.
% Cela permettra notamment de réarranger les transparents facilement
% lors de l'élaboration du document.

% La définition de la classe beamer avec tous les styles afférents
\input{preambule/special_beamer.tex}

% Les autres packages utiles  notamment pour le français, les accents ou Python
\input{preambule/autres_packages.tex}
% Les macros et raccourcis personnels
\input{preambule/macros.tex}

% On définit le titre et l'auteur du document

% L'argument optionnel (entre crochets) donne le titre qui sera mis sur chaque slide
\title[TIPE Mercure]{TIPE: Précession du Périhélie de Mercure}
\author[813]{Candidat 813} % Votre nom (pour cette année) ou numéro (pour le concours)
% L'épreuve (car on n'a pas le droit de signaler sa provenance à un concours) 
% (là encore, l'argument optionnel apparaît sur chaque slide)
\institute[TIPE]{Épreuve de TIPE}
\date{Session 2023}

% On démarre le document proprement dit
\begin{document}

% La page de titre et la table des matières
\input{00_titre_et_tdm.tex}

% La première grande partie: introduction du sujet
% Titre de la premiere partie
\section[Intro]{Introduction Historique}

%%%%%%%%%%%%%%%%%%%%%%%%%%%%%%%%%%%%%%%%%%%%%%%%
% Première diapo
%%%%%%%%%%%%%%%%%%%%%%%%%%%%%%%%%%%%%%%%%%%%%%%%
\begin{frame}
	\frametitle{Introduction historique}
	\framesubtitle{Position du problème}

	\begin{block}{Un peu d'histoire}
		La mécanique newtonienne, une mécanique bien huilée
	\end{block}

	\pause

	\begin{alertblock}{Le problème du périhélie de Mercure}
		\pause
		\begin{figure}
			\visible<+->{
				\includegraphics[width=0.6\linewidth]{figures/Fig01}
				\caption{Précession du périhélie de Mercure}
			}
		\end{figure}
	\end{alertblock}

\end{frame}


%%%%%%%%%%%%%%%%%%%%%%%%%%%%%%%%%%%%%%%%%%%%%%%%
% Deuxième diapo
%%%%%%%%%%%%%%%%%%%%%%%%%%%%%%%%%%%%%%%%%%%%%%%%
\begin{frame}
	\frametitle{Introduction historique}
	\framesubtitle{Explication classique}

	\begin{columns}
		\column{0.5\linewidth} % première colonne
			Plusieurs explications en compétitions:

			\pause

			\begin{itemize}[<+->]
				\item		Théorie des perturbations: explication de $531''$ d'arc/siècle
				\item		Vulcain, un bon candidat (Le Verrier, 1859)
				\item		mais pas de confirmation expérimentale...

			\end{itemize}



		\column{0.5\linewidth} % 2e colonne
			\begin{alertblock}{Ça ne va pas}<+->
				Il manque un élément à la théorie!
			\end{alertblock}

	\end{columns}

\end{frame}


%%%%%%%%%%%%%%%%%%%%%%%%%%%%%%%%%%%%%%%%%%%%%%%%
% Troisième diapo
%%%%%%%%%%%%%%%%%%%%%%%%%%%%%%%%%%%%%%%%%%%%%%%%
\begin{frame}
	\frametitle{Introduction historique}
	\framesubtitle{Complément relativiste}

	\begin{block}{L'avis de la relativité restreinte}
		$7''$ d'arc
		en plus par siècle
	\end{block}

	\pause

	\begin{exampleblock}{La générale sauve la mise}
		Rajoute
		les $43''$ qui manquaient.
	\end{exampleblock}

\end{frame}


% La 2e partie: Le point de vue de la relativité restreinte
% Titre de la partie
\section[Relativité restreinte]{Le point de vue de la Relativité Restreinte}

%%%%%%%%%%%%%%%%%%%%%%%%%%%%%%%%%%%%%%%%%%%%%%%%
% Première diapo (avec des équations)
%%%%%%%%%%%%%%%%%%%%%%%%%%%%%%%%%%%%%%%%%%%%%%%%
\begin{frame}
	\frametitle{Relativité Restreinte}
	\framesubtitle{Dynamique relativiste}

	\begin{block}{Nouvelle définition de la quantité de mouvement}
        \pause
		$$
		\vec{p} = \gamma\,m\vec{v}
		\qquad\text{avec}\qquad
		\gamma = \frac{1}{\sqrt{1 - \frac{v^2}{c^2}}}
		$$
	\end{block}

    \pause
		Avec des développements classique et
		en posant $u=1/r$, on en arrive à l'équation
		\onslide <4->{
		$$
		\onslide <5->
		\underbrace{
		\onslide <4->
		    \frac{\dd^2 u}{\dd \theta^2} + u = \frac{G M m\, E}{L^2\, c^2}
		\onslide <5->
		    }_{
		    \text{Partie usuelle}}
		\onslide <4->
		            +
		\onslide <6->
		\underbrace{
		\onslide <4->
		            \frac{\pa{G M m}^2}{L^2\, c^2}\, u
	    \onslide <6->
	               }_{\text{Partie relativiste}}
		$$
		}

\end{frame}


%%%%%%%%%%%%%%%%%%%%%%%%%%%%%%%%%%%%%%%%%%%%%%%%
% Deuxième diapo
%%%%%%%%%%%%%%%%%%%%%%%%%%%%%%%%%%%%%%%%%%%%%%%%
\begin{frame}
	\frametitle{Relativité Restreinte}
	\framesubtitle{Équation de l'ellipse}

	Équation différentielle remise en forme

    \pause
		$$
		    \frac{\dd u}{\dd \theta} + B^2\, u = A
            \pause
		        \quad \text{avec} \quad
		            B  = \sqrt{1 - \pa{\frac{G M m}{L\, c}}^{\!\!2} }
		$$

    \pause

	\begin{block}{Équation de l'\ofg{ellipse}}
		$$
		u = \frac{A}{B^2} \pa{1 + e\cos{\pac{B\pa{\theta - \theta_0}}}}
        \pause
		\quad\text{soit}\quad
		r = \frac{p}{1+e\cos{\pac{B\pa{\theta - \theta_0}}}}
		$$
	\end{block}

\end{frame}

%%%%%%%%%%%%%%%%%%%%%%%%%%%%%%%%%%%%%%%%%%%%%%%%
% Troisième diapo
%%%%%%%%%%%%%%%%%%%%%%%%%%%%%%%%%%%%%%%%%%%%%%%%
\begin{frame}
	\frametitle{Relativité Restreinte}
	\framesubtitle{Avance du périhélie}

	\begin{exampleblock}{Caractère non fermé}<+->
		Du fait que $B\neq1$.
	\end{exampleblock}

	\begin{exampleblock}{Rotation}<+->
		Entre deux périhélies successifs, $\theta$ tourne de $2\pi +\delta$ où
		\onslide<+->{
			$$
			\boxed{
		    \delta = 2\pi\pa{\frac{1}{B} - 1}
		    			\approx \pi \pa{\frac{G\, M\, m}{L\, c}}^{\!\!2}
		    			= \pi \, \frac{G M}{p\, c^2}
		    			= \pi\, \frac{G M}{a\, c^2\, \pa{1-e^2}}
		    }
			$$
		}
	\end{exampleblock}

	\begin{alertblock}{Malheureusement...}<+->
		l'application numérique ne donne \ofg{que}
		$7''$ d'arc par siècle...
	\end{alertblock}


\end{frame}


% La 3e partie: Le point de vue de la relativité générale
% Le titre de la partie
\section[Relativité générale]{Le point de vue de la Relativité Générale}

%%%%%%%%%%%%%%%%%%%%%%%%%%%%%%%%%%%%%%%%%%%%%%%%
% Première diapo
%%%%%%%%%%%%%%%%%%%%%%%%%%%%%%%%%%%%%%%%%%%%%%%%

\begin{frame}
	\frametitle{Relativité Générale}
	\framesubtitle{Et zut...}

	\begin{alertblock}{Arg...}
		Il n'y a plus guère de temps pour parler...

		Pour voir le code informatique, cliquez sur le bouton \beamergotobutton{\hyperlink{code}{Source}}
	\end{alertblock}

	Alors zou, on balance le résultat des calculs
	$$
	    \onslide<3->
	    \underbrace{
	    \onslide<2->
	    \frac{\dd^2 u}{\dd \theta^2} + u
	        =
	        \frac{G M m^2}{L^2}
	    \onslide<3->
	    }_{\text{Partie classique}}
	    \onslide<2->
	    +
	    \onslide<3->
	    \underbrace{
	    \onslide<2->
	    \frac{3\, G M}{c^2}\, u^2
	    \onslide<3->
	    }_{\text{RG}}
	    \onslide<2->
	$$
	\onslide<+->{
	Après avoir bien bossé, on obtient finalement
	$$
		\boxed{
			\delta = 6\, \pi \, \frac{G M}{a\, c^2\pa{1-e^2}}
				}
	$$
	}
\end{frame}


% Conclusion
% Le titre de la partie
\section{Conclusion}

%%%%%%%%%%%%%%%%%%%%%%%%%%%%%%%%%%%%%%%%%%%%%%%%
% Première diapo avec un exemple de tableau
%%%%%%%%%%%%%%%%%%%%%%%%%%%%%%%%%%%%%%%%%%%%%%%%
\begin{frame}
\frametitle{Conclusion}
\framesubtitle{sous forme de tableau}

\begin{table}
\begin{tabular}{c c c}
\toprule
\textbf{Newton} & \textbf{Rel. Restreinte} & \textbf{Rel. Générale}\\
\midrule
$531''/$siècle & ($+7''/$siècle) & $+43''/$siècle \\
\midrule
\multicolumn{3}{c}{Observations: $574''/$siècle} \\
\bottomrule
\end{tabular}
\caption{Effet des différentes théories}
\end{table}

\end{frame}



% Annexes (non numérotées)
\appendix
%\backupbegin

%%%%%%%%%%%%%%%%%%%%%%%%%%%%%%%%%%%%%%%%%%%%%%%%
% Le code informatique impose un
% environnement "fragile" pour la frame
%%%%%%%%%%%%%%%%%%%%%%%%%%%%%%%%%%%%%%%%%%%%%%%%

\begin{frame}[fragile, label=code]
\frametitle{Relativité Générale}
\framesubtitle{Le code informatique}

\begin{code}
\begin{minted}[linenos]{python}

import scipy as sp
import scipy.optimize

def ma_fonction(x):
    return []

# À vous de remplir les choses adéquates...

\end{minted}
\end{code}
\end{frame}



% Les diapos d'exemples
% (à commenter si bien sûr vous n'en voulez pas...
% ils sont juste là pour servir d'exemples de base)
\section[Exemples]{Exemples divers}
\begin{frame}
\frametitle{Exemples}
\framesubtitle{Apparitions successives}

\begin{itemize}
	\item	<1->	Ce point apparaît en premier et reste tout le temps
	\item	<2>	    Celui-ci ne n'apparaîtra que à la 2\ieme{} page de cette diapo (mais l'espace reste disponible)
	\item	<4->	...avant donc l'apparition du 4\ieme{} (mais c'est bizarre de procéder ainsi)
	\item	<3->	Et celui-ci vient en 3\ieme{} et reste jusqu'à la fin...
\end{itemize}

\end{frame}

\begin{frame}
\frametitle{Exemples}
\framesubtitle{Apparition d'une figure}

On peut aussi mettre du texte brut avant apparition d'une figure

\visible<2-> {% On n'utilise pas "\onslide" ici car on ne peut pas régler le niveau de transparence de la figure
\begin{center}
\includegraphics[width=0.5\linewidth]{figures/Fig02}
\end{center}
} % NB: \visible rend visible son argument, c'est-à-dire qu'il faut mettre ce 
  % qu'on veut rendre visible entre accolades par la suite contrairement à \onslide
  % qui joue le rôle d'une bascule.

\onslide<3->
Et le texte qui suit

\end{frame}

\begin{frame}[fragile]
\frametitle{Exemples}
\framesubtitle{Apparition d'une équation en plusieurs temps}

L'idée est d'utiliser \verb|\onslide| pour faire apparaître les morceaux uns à 
uns. Cela correspond à des bascules qui imposent le comportement de tout ce qui 
suit jusqu'au prochain \verb|\onslide|. Par exemple, avec votre vieil ami 
l'oscillateur harmonique

$$
\onslide<3>
	1\times
\onslide<2->
	\frac{\dd^2\theta}{\dd t^2} + 
\onslide<4>
	\overbrace{
\onslide<2-> 
	\frac{g}{\ell}
\onslide<4>
	}^{={\omega_0}^2}
\onslide<2-> 
	\times\theta
	=
\onslide<5>
	\overbrace{
\onslide<2-> 
	A
\onslide<5>
	}^{=\theta\e{éq}\times{\omega_0}^2}
\onslide<2-> 
$$

\onslide<6->
Le mieux est d'écrire l'équation voulue en une fois avec tous les rajouts et de découper ensuite. Par exemple ici, ce serait

\onslide<7->
$$
	1\times
	\frac{\dd^2\theta}{\dd t^2} + 
	\overbrace{  	\frac{g}{\ell}  	}^{={\omega_0}^2}
	\times\theta	=
	\overbrace{	A	}^{=\theta\e{éq}\times{\omega_0}^2}
$$


\end{frame}




%\backupend

\end{document}
